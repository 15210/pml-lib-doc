\chapter{Table Interface}
\label{ch:table-interface}
\begin{preamble}
The \sml{TABLE} interface specifies a mapping from keys to values, written
$\{k_1 \mapsto v_1, k_2 \mapsto v_2, \ldots\}$. Tables do not have duplicate
keys, so there is a unique associated value for each key in the domain of
a table. The key type is given by the \sml{Key} substructure, and the value
type is polymorphic.
\end{preamble}

\section{Summary}
\begin{gram}
\begin{verbatim}
signature TABLE =
sig
  structure Key : EQKEY
  structure Seq : SEQUENCE

  type 'a t
  type 'a table = 'a t

  structure Set : SET where Key = Key and Seq = Seq

  val size : 'a table -> int
  val domain : 'a table -> Set.t
  val range : 'a table -> 'a Seq.t
  val toString : ('a -> string) -> 'a table -> string
  val toSeq : 'a table -> (Key.t * 'a) Seq.t

  val find : 'a table -> Key.t -> 'a option
  val insert : 'a table * (Key.t * 'a) -> 'a table
  val insertWith : ('a * 'a -> 'a) -> 'a table * (Key.t * 'a) -> 'a table
  val delete : 'a table * Key.t -> 'a table

  val empty : unit -> 'a table
  val singleton : Key.t * 'a -> 'a table
  val tabulate : (Key.t -> 'a) -> Set.t -> 'a table
  val collect : (Key.t * 'a) Seq.t -> 'a Seq.t table
  val fromSeq : (Key.t * 'a) Seq.t -> 'a table

  val map : ('a -> 'b) -> 'a table -> 'b table
  val mapKey : (Key.t * 'a -> 'b) -> 'a table -> 'b table
  val filter : ('a -> bool) -> 'a table -> 'a table
  val filterKey : (Key.t * 'a -> bool) -> 'a table -> 'a table

  val reduce : ('a * 'a -> 'a) -> 'a -> 'a table -> 'a
  val iterate : ('b * 'a -> 'b) -> 'b -> 'a table -> 'b
  val iteratePrefixes : ('b * 'a -> 'b) -> 'b -> 'a table -> ('b table * 'b)

  val union : ('a * 'a -> 'a) -> ('a table * 'a table) -> 'a table
  val intersection : ('a * 'b -> 'c) -> 'a table * 'b table -> 'c table
  val difference : 'a table * 'b table -> 'a table

  val restrict : 'a table * Set.t -> 'a table
  val subtract : 'a table * Set.t -> 'a table

  val $ : (Key.t * 'a) -> 'a table
end
\end{verbatim}
\end{gram}

\section{Substructures}

\begin{gram}
\begin{verbatim}
structure Key : EQKEY
\end{verbatim}
The \sml{Key} substructure defines the type of keys in a table, which may be
compared for equality.
\end{gram}

\begin{gram}
\begin{verbatim}
structure Seq : SEQUENCE
\end{verbatim}
The \sml{Seq} substructure defines the underlying sequence type, so that we
may convert tables to and from sequences.
\end{gram}

\begin{gram}
\begin{verbatim}
structure Set : SET
\end{verbatim}
The \sml{Set} substructure contains operations on sets with elements of type
\sml{Key.t}.
\end{gram}


\section{Types}

\begin{gram}
\begin{verbatim}
type 'a t
type 'a table = 'a t
\end{verbatim}
The abstract table type with values of type \sml{'a}. The alias \sml{table} is
for readability in the signature.
\end{gram}

\section{Values}

\begin{gram}[size]
\begin{verbatim}
val size : 'a table -> int
\end{verbatim}
The number of key-value pairs in a table.
\end{gram}

\begin{gram}[domain]
\begin{verbatim}
val domain : 'a table -> Set.t
\end{verbatim}
Return the set of all keys in a table.
\end{gram}

\begin{gram}[range]
\begin{verbatim}
val range : 'a table -> 'a Seq.t
\end{verbatim}
Return a sequence of all values in a table. The order of the elements is
implementation-defined.
\end{gram}

\begin{gram}[toString]
\begin{verbatim}
val toString : ('a -> string) -> 'a table -> string
\end{verbatim}
\sml{toString f t} returns a string representation of $t$. Each key is converted
to a string via \sml{Key.toString} and each value is converted via $f$.
\end{gram}

\begin{gram}[toSeq]
\begin{verbatim}
val toSeq : 'a table -> (Key.t * 'a) Seq.t
\end{verbatim}
Return a sequence of all key-value pairs in a table. The order of the sequence
is implementation-defined.
\end{gram}

\begin{gram}[find]
\begin{verbatim}
val find : 'a table -> Key.t -> 'a option
\end{verbatim}
\sml{find t k} returns \sml{SOME v} if $(k \mapsto v) \in t$, and \sml{NONE}
otherwise.
\end{gram}

\begin{gram}[insert]
\begin{verbatim}
val insert : 'a table * (Key.t * 'a) -> 'a table
\end{verbatim}
\sml{insert (t, (k, v))} returns $t \cup \{k \mapsto v\}$. If $k$ is already
in $t$, then the new value $v$ is given precedence. It is logically equivalent
to \sml{insertWith (fn (\_, v) => v) (t, (k, v))}.
\end{gram}

\begin{gram}[insertWith]
\begin{verbatim}
val insertWith : ('a * 'a -> 'a) -> 'a table * (Key.t * 'a) -> 'a table
\end{verbatim}
\sml{insertWith f (t, (k, v))} returns $t \cup \{k \mapsto v\}$ if $k$ is not
already a member of $t$, and otherwise it returns $t \cup \{k \mapsto f(v',v)\}$
where $k \mapsto v'$ is already in $t$.
\end{gram}

\begin{gram}[delete]
\begin{verbatim}
val delete : 'a table * Key.t -> 'a table
\end{verbatim}
\sml{delete (t, k)} removes the key $k$ from $t$ only if $k$ is a member of $t$.
That is, if $(k \mapsto v) \in t$ then it returns
$t \setminus \{(k \mapsto v)\}$, otherwise it returns $t$.
\end{gram}

\begin{gram}[empty]
\begin{verbatim}
val empty : unit -> 'a table
\end{verbatim}
Construct the empty table.
\end{gram}

\begin{gram}[singleton]
\label{gr:table-interface:singleton}
\begin{verbatim}
val singleton : Key.t * 'a -> 'a table
\end{verbatim}
\sml{singleton (k, v)} constructs the singleton table $\{k \mapsto v\}$.
\end{gram}

\begin{gram}[tabulate]
\begin{verbatim}
val tabulate : (Key.t -> 'a) -> Set.t -> 'a table
\end{verbatim}
\sml{tabulate f s} evaluates to the table $\{(k \mapsto f(k)) : k \in s\}$.
\end{gram}

\begin{gram}[collect]
\begin{verbatim}
val collect : (Key.t * 'a) Seq.t -> 'a Seq.t table
\end{verbatim}
\sml{collect s} takes a sequence of key-value pairs and produces a table where
each unique key $k$ is paired with
$\langle v : (k',v) \in s \mathbin| k' = k \rangle$.
\end{gram}

\begin{gram}[fromSeq]
\begin{verbatim}
val fromSeq : (Key.t * 'a) Seq.t -> 'a table
\end{verbatim}
Return the table representation of a sequence of key-value pairs. If there are
duplicate keys, then it takes the value associated with the first occurence in
the sequence.
\end{gram}

\begin{gram}[map]
\begin{verbatim}
val map : ('a -> 'b) -> 'a table -> 'b table
\end{verbatim}
\sml{map f t} evaluates to $\{k \mapsto f(v) : (k \mapsto v) \in t\}$.
\end{gram}

\begin{gram}[mapKey]
\begin{verbatim}
val mapKey : (Key.t * 'a -> 'b) -> 'a table -> 'b table
\end{verbatim}
\sml{mapKey f t} evaluates to $\{k \mapsto f(k, v) : (k \mapsto v) \in t\}$.
\end{gram}

\begin{gram}[filter]
\begin{verbatim}
val filter : ('a -> bool) -> 'a table -> 'a table
\end{verbatim}
\sml{filter p t} produces a table containing all $(k \mapsto v) \in t$ which
satisfies $p(v)$.
\end{gram}

\begin{gram}[filterKey]
\begin{verbatim}
val filterKey : (Key.t * 'a -> bool) -> 'a table -> 'a table
\end{verbatim}
\sml{filterKey p t} produces a table containing all $(k \mapsto v) \in t$ which
satisfies $p(k,v)$.
\end{gram}

\begin{gram}[reduce]
\begin{verbatim}
val reduce : ('a * 'a -> 'a) -> 'a -> 'a table -> 'a
\end{verbatim}
\sml{reduce f b t} is logically equivalent to \sml{Seq.reduce f b (range t)}.
\end{gram}

\begin{gram}[iterate]
\begin{verbatim}
val iterate : ('b * 'a -> 'b) -> 'b -> 'a table -> 'b
\end{verbatim}
\sml{iterate f b t} is logically equivalent to \sml{Seq.iterate f b (range t)}.
\end{gram}

\begin{gram}[iteratePrefixes]
\begin{verbatim}
val iteratePrefixes : ('b * 'a -> 'b) -> 'b -> 'a table -> ('b table * 'b)
\end{verbatim}
\sml{iteratePrefixes f b t} is logically equivalent to
\sml{(fromSeq p, x)} where \sml{(p,x) = Seq.iteratePrefixes f b (range t)}.
\end{gram}

\begin{gram}[union]
\begin{verbatim}
val union : ('a * 'a -> 'a) -> ('a table * 'a table) -> 'a table
\end{verbatim}
\sml{union f (a, b)} evaluates to $a \cup b$. For keys $k$ where $(k \mapsto v) \in a$
and $(k \mapsto w) \in b$, the result contains $k \mapsto f(v,w)$.
\end{gram}

\begin{gram}[intersection]
\begin{verbatim}
val intersection : ('a * 'b -> 'c) -> 'a table * 'b table -> 'c table
\end{verbatim}
\sml{intersection f (a, b)} evaluates to $a \cap b$. Every intersecting key
$k$ is mapped to $f(v,w)$ where $(k \mapsto v) \in a$ and $(k \mapsto w) \in b$.
\end{gram}

\begin{gram}[difference]
\begin{verbatim}
val difference : 'a table * 'b table -> 'a table
\end{verbatim}
\sml{difference (a, b)} evaluates to $a \setminus b$. The values in the output
are taken from $a$.
\end{gram}

\begin{gram}[restrict]
\begin{verbatim}
val restrict : 'a table * Set.t -> 'a table
\end{verbatim}
\sml{restrict (t, s)} returns the table of $\{(k \mapsto v) \in t \mathbin| k \in s\}$.
It is therefore essentially an intersection. The name is motivated by the
notion of restricting a function to a smaller domain, where we interpret a table
as a function.
\end{gram}

\begin{gram}[subtract]
\begin{verbatim}
val subtract : 'a table * Set.t -> 'a table
\end{verbatim}
\sml{subtract (t, s)} returns the table of
$\{(k \mapsto v) \in t \mathbin| k \not\in s\}$.
The name is motivated by the notion of a domain subtraction on a function.
\end{gram}

\begin{gram}[\$]
\begin{verbatim}
val $ : (Key.t * 'a) -> 'a table
\end{verbatim}
An alias for \ref{gr:table-interface:singleton}.
\end{gram}


