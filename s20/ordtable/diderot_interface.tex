\chapter{OrdTable Interface}
\label{ch:ordtable-interface}

\begin{cluster}
\label{grp:grm:ordtable-interface::ordtable}

\begin{gram}
\label{grm:ordtable-interface::ordtable}
\label{ch:ordtable-interface}
\begin{preamble}
The \sml{ORDTABLE} interface specifies a mapping from keys to values, written
$\{k_1 \mapsto v_1, k_2 \mapsto v_2, \ldots\}$. Tables do not have duplicate
keys, so there is a unique associated value for each key in the domain of
a table. The key type is given by the \sml{Key} substructure, and the value
type is polymorphic.
\end{preamble}

\end{gram}
\end{cluster}

\begin{cluster}
\label{grp:nt:ordtable-interface::ordtable}

\begin{note}
\label{nt:ordtable-interface::ordtable}
The \sml{ORDTABLE} signature is identical to \sml{TABLE} except for the following.
\begin{itemize}
  \item The \sml{Key} substructure now ascribes to \sml{ORDKEY}.
  \item There is additional functionality relying upon the ordering of keys,
  such as \sml{split}, \sml{join}, and \sml{getRange}.
  \item The \sml{toSeq} function now specifies that it returns keys in sorted order.
\end{itemize}

\end{note}
\end{cluster}


\section{Summary}
\label{sec:ordtable-interface::summary}

\begin{cluster}
\label{grp:grm:ordtable-interface::signature}

\begin{gram}
\label{grm:ordtable-interface::signature}
\begin{verbatim}
signature ORDTABLE =
sig
  structure Key : ORDKEY
  structure Seq : SEQUENCE

  type 'a t
  type 'a table = 'a t

  structure Set : ORDSET where Key = Key and Seq = Seq

  exception Order

  val size : 'a table -> int
  val domain : 'a table -> Set.t
  val range : 'a table -> 'a Seq.t
  val toString : ('a -> string) -> 'a table -> string
  val toSeq : 'a table -> (Key.t * 'a) Seq.t

  val find : 'a table -> Key.t -> 'a option
  val insert : 'a table * (Key.t * 'a) -> 'a table
  val insertWith : ('a * 'a -> 'a) -> 'a table * (Key.t * 'a) -> 'a table
  val delete : 'a table * Key.t -> 'a table

  val empty : unit -> 'a table
  val singleton : Key.t * 'a -> 'a table
  val tabulate : (Key.t -> 'a) -> Set.t -> 'a table
  val collect : (Key.t * 'a) Seq.t -> 'a Seq.t table
  val fromSeq : (Key.t * 'a) Seq.t -> 'a table

  val map : ('a -> 'b) -> 'a table -> 'b table
  val mapKey : (Key.t * 'a -> 'b) -> 'a table -> 'b table
  val filter : ('a -> bool) -> 'a table -> 'a table
  val filterKey : (Key.t * 'a -> bool) -> 'a table -> 'a table

  val reduce : ('a * 'a -> 'a) -> 'a -> 'a table -> 'a
  val iterate : ('b * 'a -> 'b) -> 'b -> 'a table -> 'b
  val iteratePrefixes : ('b * 'a -> 'b) -> 'b -> 'a table -> ('b table * 'b)

  val union : ('a * 'a -> 'a) -> ('a table * 'a table) -> 'a table
  val intersection : ('a * 'b -> 'c) -> 'a table * 'b table -> 'c table
  val difference : 'a table * 'b table -> 'a table

  val restrict : 'a table * Set.t -> 'a table
  val subtract : 'a table * Set.t -> 'a table

  val $ : (Key.t * 'a) -> 'a table

  (* ordered tables begins here *)
  val first : 'a table -> (Key.t * 'a) option
  val last : 'a table -> (Key.t * 'a) option

  val prev : 'a table * Key.t -> (Key.t * 'a) option
  val next : 'a table * Key.t -> (Key.t * 'a) option

  val split : 'a table * Key.t -> 'a table * 'a option * 'a table
  val join : 'a table * 'a table -> 'a table

  val getRange : 'a table -> Key.t * Key.t -> 'a table

  val rank : 'a table * Key.t -> int
  val select : 'a table * int -> (Key.t * 'a) option
  val splitRank : 'a table * int -> 'a table * 'a table
end
\end{verbatim}

\end{gram}
\end{cluster}


\section{Substructures}
\label{sec:ordtable-interface::substructures}

\begin{cluster}
\label{grp:grm:ordtable-interface::structure}

\begin{gram}
\label{grm:ordtable-interface::structure}
\begin{verbatim}
structure Key : ORDKEY
\end{verbatim}
The \sml{Key} substructure defines the type of keys in a table, which are
totally ordered according to the provided comparison function.

\end{gram}
\end{cluster}

\begin{cluster}
\label{grp:grm:ordtable-interface::sequence}

\begin{gram}
\label{grm:ordtable-interface::sequence}
\begin{verbatim}
structure Seq : SEQUENCE
\end{verbatim}
The \sml{Seq} substructure defines the underlying sequence type, so that we
may convert tables to and from sequences.

\end{gram}
\end{cluster}

\begin{cluster}
\label{grp:grm:ordtable-interface::ordset}

\begin{gram}
\label{grm:ordtable-interface::ordset}
\begin{verbatim}
structure Set : ORDSET
\end{verbatim}
The \sml{Set} substructure contains operations on ordered sets with elements of
type \sml{Key.t}.

\end{gram}
\end{cluster}


\section{Types}
\label{sec:ordtable-interface::types}

\begin{cluster}
\label{grp:grm:ordtable-interface::type}

\begin{gram}
\label{grm:ordtable-interface::type}
\begin{verbatim}
type 'a t
type 'a table = 'a t
\end{verbatim}
The abstract table type with values of type \sml{'a}. The alias \sml{table} is
for readability in the signature.

\end{gram}
\end{cluster}


\section{Exceptions}
\label{sec:ordtable-interface::exceptions}

\begin{cluster}
\label{grp:grm:ordtable-interface::exception}

\begin{gram}
\label{grm:ordtable-interface::exception}
\begin{verbatim}
exception Order
\end{verbatim}
\sml{Order} is raised when the ordering invariant would be violated.

\end{gram}
\end{cluster}


\section{Values}
\label{sec:ordtable-interface::values}

\begin{cluster}
\label{grp:grm:ordtable-interface::size}

\begin{gram}[size]
\label{grm:ordtable-interface::size}
\begin{verbatim}
val size : 'a table -> int
\end{verbatim}
The number of key-value pairs in a table.

\end{gram}
\end{cluster}

\begin{cluster}
\label{grp:grm:ordtable-interface::domain}

\begin{gram}[domain]
\label{grm:ordtable-interface::domain}
\begin{verbatim}
val domain : 'a table -> Set.t
\end{verbatim}
Return the set of all keys in a table.

\end{gram}
\end{cluster}

\begin{cluster}
\label{grp:grm:ordtable-interface::range}

\begin{gram}[range]
\label{grm:ordtable-interface::range}
\begin{verbatim}
val range : 'a table -> 'a Seq.t
\end{verbatim}
Return a sequence of all values in a table. The order of the elements is
implementation-defined.

\end{gram}
\end{cluster}

\begin{cluster}
\label{grp:grm:ordtable-interface::tostring}

\begin{gram}[toString]
\label{grm:ordtable-interface::tostring}
\begin{verbatim}
val toString : ('a -> string) -> 'a table -> string
\end{verbatim}
\sml{toString f t} returns a string representation of $t$. Each key is converted
to a string via \sml{Key.toString} and each value is converted via $f$.

\end{gram}
\end{cluster}

\begin{cluster}
\label{grp:grm:ordtable-interface::toseq}

\begin{gram}[toSeq]
\label{grm:ordtable-interface::toseq}
\begin{verbatim}
val toSeq : 'a table -> (Key.t * 'a) Seq.t
\end{verbatim}
Return a sequence of all key-value pairs in a table, sorted by key.

\end{gram}
\end{cluster}

\begin{cluster}
\label{grp:grm:ordtable-interface::find}

\begin{gram}[find]
\label{grm:ordtable-interface::find}
\begin{verbatim}
val find : 'a table -> Key.t -> 'a option
\end{verbatim}
\sml{find t k} returns \sml{SOME v} if $(k \mapsto v) \in t$, and \sml{NONE}
otherwise.

\end{gram}
\end{cluster}

\begin{cluster}
\label{grp:grm:ordtable-interface::insert}

\begin{gram}[insert]
\label{grm:ordtable-interface::insert}
\begin{verbatim}
val insert : 'a table * (Key.t * 'a) -> 'a table
\end{verbatim}
\sml{insert (t, (k, v))} returns $t \cup \{k \mapsto v\}$. If $k$ is already
in $t$, then the new value $v$ is given precedence. It is logically equivalent
to \sml{insertWith (fn (\_, v) => v) (t, (k, v))}.

\end{gram}
\end{cluster}

\begin{cluster}
\label{grp:grm:ordtable-interface::insertwith}

\begin{gram}[insertWith]
\label{grm:ordtable-interface::insertwith}
\begin{verbatim}
val insertWith : ('a * 'a -> 'a) -> 'a table * (Key.t * 'a) -> 'a table
\end{verbatim}
\sml{insertWith f (t, (k, v))} returns $t \cup \{k \mapsto v\}$ if $k$ is not
already a member of $t$, and otherwise it returns $t \cup \{k \mapsto f(v',v)\}$
where $k \mapsto v'$ is already in $t$.

\end{gram}
\end{cluster}

\begin{cluster}
\label{grp:grm:ordtable-interface::delete}

\begin{gram}[delete]
\label{grm:ordtable-interface::delete}
\begin{verbatim}
val delete : 'a table * Key.t -> 'a table
\end{verbatim}
\sml{delete (t, k)} removes the key $k$ from $t$ only if $k$ is a member of $t$.
That is, if $(k \mapsto v) \in t$ then it returns
$t \setminus \{(k \mapsto v)\}$, otherwise it returns $t$.

\end{gram}
\end{cluster}

\begin{cluster}
\label{grp:grm:ordtable-interface::empty}

\begin{gram}[empty]
\label{grm:ordtable-interface::empty}
\begin{verbatim}
val empty : unit -> 'a table
\end{verbatim}
Construct the empty table.

\end{gram}
\end{cluster}

\begin{cluster}
\label{grp:gr:ordtable-interface:singleton}

\begin{gram}[singleton]
\label{gr:ordtable-interface:singleton}
\begin{verbatim}
val singleton : Key.t * 'a -> 'a table
\end{verbatim}
\sml{singleton (k, v)} constructs the singleton table $\{k \mapsto v\}$.

\end{gram}
\end{cluster}

\begin{cluster}
\label{grp:grm:ordtable-interface::tabulate}

\begin{gram}[tabulate]
\label{grm:ordtable-interface::tabulate}
\begin{verbatim}
val tabulate : (Key.t -> 'a) -> Set.t -> 'a table
\end{verbatim}
\sml{tabulate f s} evaluates to the table $\{(k \mapsto f(k)) : k \in s\}$.

\end{gram}
\end{cluster}

\begin{cluster}
\label{grp:grm:ordtable-interface::collect}

\begin{gram}[collect]
\label{grm:ordtable-interface::collect}
\begin{verbatim}
val collect : (Key.t * 'a) Seq.t -> 'a Seq.t table
\end{verbatim}
\sml{collect s} takes a sequence of key-value pairs and produces a table where
each unique key $k$ is paired with
$\langle v : (k',v) \in s \mathbin| k' = k \rangle$.

\end{gram}
\end{cluster}

\begin{cluster}
\label{grp:grm:ordtable-interface::fromseq}

\begin{gram}[fromSeq]
\label{grm:ordtable-interface::fromseq}
\begin{verbatim}
val fromSeq : (Key.t * 'a) Seq.t -> 'a table
\end{verbatim}
Return the table representation of a sequence of key-value pairs. If there are
duplicate keys, then it takes the value associated with the first occurence in
the sequence.

\end{gram}
\end{cluster}

\begin{cluster}
\label{grp:grm:ordtable-interface::map}

\begin{gram}[map]
\label{grm:ordtable-interface::map}
\begin{verbatim}
val map : ('a -> 'b) -> 'a table -> 'b table
\end{verbatim}
\sml{map f t} evaluates to $\{k \mapsto f(v) : (k \mapsto v) \in t\}$.

\end{gram}
\end{cluster}

\begin{cluster}
\label{grp:grm:ordtable-interface::mapkey}

\begin{gram}[mapKey]
\label{grm:ordtable-interface::mapkey}
\begin{verbatim}
val mapKey : (Key.t * 'a -> 'b) -> 'a table -> 'b table
\end{verbatim}
\sml{mapKey f t} evaluates to $\{k \mapsto f(k, v) : (k \mapsto v) \in t\}$.

\end{gram}
\end{cluster}

\begin{cluster}
\label{grp:grm:ordtable-interface::filter}

\begin{gram}[filter]
\label{grm:ordtable-interface::filter}
\begin{verbatim}
val filter : ('a -> bool) -> 'a table -> 'a table
\end{verbatim}
\sml{filter p t} produces a table containing all $(k \mapsto v) \in t$ which
satisfies $p(v)$.

\end{gram}
\end{cluster}

\begin{cluster}
\label{grp:grm:ordtable-interface::filterkey}

\begin{gram}[filterKey]
\label{grm:ordtable-interface::filterkey}
\begin{verbatim}
val filterKey : (Key.t * 'a -> bool) -> 'a table -> 'a table
\end{verbatim}
\sml{filterKey p t} produces a table containing all $(k \mapsto v) \in t$ which
satisfies $p(k,v)$.

\end{gram}
\end{cluster}

\begin{cluster}
\label{grp:grm:ordtable-interface::reduce}

\begin{gram}[reduce]
\label{grm:ordtable-interface::reduce}
\begin{verbatim}
val reduce : ('a * 'a -> 'a) -> 'a -> 'a table -> 'a
\end{verbatim}
\sml{reduce f b t} is logically equivalent to \sml{Seq.reduce f b (range t)}.

\end{gram}
\end{cluster}

\begin{cluster}
\label{grp:grm:ordtable-interface::iterate}

\begin{gram}[iterate]
\label{grm:ordtable-interface::iterate}
\begin{verbatim}
val iterate : ('b * 'a -> 'b) -> 'b -> 'a table -> 'b
\end{verbatim}
\sml{iterate f b t} is logically equivalent to \sml{Seq.iterate f b (range t)}.

\end{gram}
\end{cluster}

\begin{cluster}
\label{grp:grm:ordtable-interface::iterateprefixes}

\begin{gram}[iteratePrefixes]
\label{grm:ordtable-interface::iterateprefixes}
\begin{verbatim}
val iteratePrefixes : ('b * 'a -> 'b) -> 'b -> 'a table -> ('b table * 'b)
\end{verbatim}
\sml{iteratePrefixes f b t} is logically equivalent to
\sml{(fromSeq p, x)} where \sml{(p,x) = Seq.iteratePrefixes f b (range t)}.

\end{gram}
\end{cluster}

\begin{cluster}
\label{grp:grm:ordtable-interface::union}

\begin{gram}[union]
\label{grm:ordtable-interface::union}
\begin{verbatim}
val union : ('a * 'a -> 'a) -> ('a table * 'a table) -> 'a table
\end{verbatim}
\sml{union f (a, b)} evaluates to $a \cup b$. For keys $k$ where $(k \mapsto v) \in a$
and $(k \mapsto w) \in b$, the result contains $k \mapsto f(v,w)$.

\end{gram}
\end{cluster}

\begin{cluster}
\label{grp:grm:ordtable-interface::intersection}

\begin{gram}[intersection]
\label{grm:ordtable-interface::intersection}
\begin{verbatim}
val intersection : ('a * 'b -> 'c) -> 'a table * 'b table -> 'c table
\end{verbatim}
\sml{intersection f (a, b)} evaluates to $a \cap b$. Every intersecting key
$k$ is mapped to $f(v,w)$ where $(k \mapsto v) \in a$ and $(k \mapsto w) \in b$.

\end{gram}
\end{cluster}

\begin{cluster}
\label{grp:grm:ordtable-interface::difference}

\begin{gram}[difference]
\label{grm:ordtable-interface::difference}
\begin{verbatim}
val difference : 'a table * 'b table -> 'a table
\end{verbatim}
\sml{difference (a, b)} evaluates to $a \setminus b$. The values in the output
are taken from $a$.

\end{gram}
\end{cluster}

\begin{cluster}
\label{grp:grm:ordtable-interface::restrict}

\begin{gram}[restrict]
\label{grm:ordtable-interface::restrict}
\begin{verbatim}
val restrict : 'a table * Set.t -> 'a table
\end{verbatim}
\sml{restrict (t, s)} returns the table of $\{(k \mapsto v) \in t \mathbin| k \in s\}$.
It is therefore essentially an intersection. The name is motivated by the
notion of restricting a function to a smaller domain, where we interpret a table
as a function.

\end{gram}
\end{cluster}

\begin{cluster}
\label{grp:grm:ordtable-interface::subtract}

\begin{gram}[subtract]
\label{grm:ordtable-interface::subtract}
\begin{verbatim}
val subtract : 'a table * Set.t -> 'a table
\end{verbatim}
\sml{subtract (t, s)} returns the table of
$\{(k \mapsto v) \in t \mathbin| k \not\in s\}$.
The name is motivated by the notion of a domain subtraction on a function.

\end{gram}
\end{cluster}

\begin{cluster}
\label{grp:grm:ordtable-interface::table}

\begin{gram}[\$]
\label{grm:ordtable-interface::table}
\begin{verbatim}
val $ : (Key.t * 'a) -> 'a table
\end{verbatim}
An alias for \ref{gr:ordtable-interface:singleton}.

\end{gram}
\end{cluster}

\begin{cluster}
\label{grp:grm:ordtable-interface::first}

\begin{gram}[first]
\label{grm:ordtable-interface::first}
\begin{verbatim}
val first : 'a table -> (Key.t * 'a) option
\end{verbatim}
Return the smallest key (together with its associated value) of a table, or
\sml{NONE} if the table is empty.

\end{gram}
\end{cluster}

\begin{cluster}
\label{grp:grm:ordtable-interface::last}

\begin{gram}[last]
\label{grm:ordtable-interface::last}
\begin{verbatim}
val last : 'a table -> (Key.t * 'a) option
\end{verbatim}
Return the largest key (together with its associated value) of a table, or
\sml{NONE} if the table is empty.

\end{gram}
\end{cluster}

\begin{cluster}
\label{grp:grm:ordtable-interface::prev}

\begin{gram}[prev]
\label{grm:ordtable-interface::prev}
\begin{verbatim}
val prev : 'a table * Key.t -> (Key.t * 'a) option
\end{verbatim}
\sml{prev (t, k)} returns the largest key (together with its associated value)
in $t$
which is strictly smaller than $k$, or \sml{NONE} if there is no such key.

\end{gram}
\end{cluster}

\begin{cluster}
\label{grp:grm:ordtable-interface::next}

\begin{gram}[next]
\label{grm:ordtable-interface::next}
\begin{verbatim}
val next : 'a table * Key.t -> (Key.t * 'a) option
\end{verbatim}
\sml{next (t, k)} returns the smallest key (together with its associated value)
in $t$
which is strictly greater than $k$, or \sml{NONE} if there is no such key.

\end{gram}
\end{cluster}

\begin{cluster}
\label{grp:grm:ordtable-interface::split}

\begin{gram}[split]
\label{grm:ordtable-interface::split}
\begin{verbatim}
val split : 'a table * Key.t -> 'a table * 'a option * 'a table
\end{verbatim}
\sml{split (t, k)} evaluates to $(\ell, v, r)$ where $\ell$ consists of all
key-value pairs with keys strictly smaller than $k$, $r$ consists of all key-value
pairs with keys strictly greater than $k$, and $v$ is $k$'s associated value
in $t$ (or \sml{NONE} if $k$ is not present in $t$).

\end{gram}
\end{cluster}

\begin{cluster}
\label{grp:grm:ordtable-interface::join}

\begin{gram}[join]
\label{grm:ordtable-interface::join}
\begin{verbatim}
val join : 'a table * 'a table -> 'a table
\end{verbatim}
Given tables $a$ and $b$ where all keys in $a$ are strictly smaller than all
keys in $b$, \sml{join (a, b)} evaluates to $a \cup b$. Otherwise, it raises
\sml{Order}.

\end{gram}
\end{cluster}

\begin{cluster}
\label{grp:grm:ordtable-interface::getrange}

\begin{gram}[getRange]
\label{grm:ordtable-interface::getrange}
\begin{verbatim}
val getRange : 'a table -> Key.t * Key.t -> 'a table
\end{verbatim}
\sml{getRange t (x, y)} evaluates to $\{(k \mapsto v) \in t \mathbin| x \leq k \leq y\}$

\end{gram}
\end{cluster}

\begin{cluster}
\label{grp:grm:ordtable-interface::rank}

\begin{gram}[rank]
\label{grm:ordtable-interface::rank}
\begin{verbatim}
val rank : 'a table * Key.t -> int
\end{verbatim}
\sml{rank (t, k)} evaluates to the number of keys in $t$ which are strictly
smaller than $k$.

\end{gram}
\end{cluster}

\begin{cluster}
\label{grp:grm:ordtable-interface::select}

\begin{gram}[select]
\label{grm:ordtable-interface::select}
\begin{verbatim}
val select : 'a table * int -> (Key.t * 'a) option
\end{verbatim}
\sml{select (t, i)} evaluates to the $i^\text{th}$ smallest key (and its
associated value), or \sml{NONE} if either $i < 0$ or $i \geq |t|$.

\end{gram}
\end{cluster}

\begin{cluster}
\label{grp:grm:ordtable-interface::splitrank}

\begin{gram}[splitRank]
\label{grm:ordtable-interface::splitrank}
\begin{verbatim}
val splitRank : 'a table * int -> 'a table * 'a table
\end{verbatim}
\sml{splitRank (t, i)} evalutes to $(\ell,r)$ where $\ell$ consists of the $i$
keys of $t$, and $r$ is the set of the $|t|-i$ largest
keys of $t$. Raises \sml{Fail} if either $i < 0$ or $i \geq |t|$.

\end{gram}
\end{cluster}

